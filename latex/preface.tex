\chapter{前言}
\label{前言}

回馈开源

\section{为什么写这本书}
\label{为什么写这本书}

omnetppp-zh.pdf记录了我在设计无人机蜂群网络仿真过程中,从初学OMNeT++软件到能灵活使用各种接口所遇到的各种问题,苦于当初无处找到详细的OMNeT++工程开发资料,尤其是针对实际功能实现的代码说明资料基本没有。我在阅读大量的网络仿真程序后,慢慢的对这个软件的各种接口和配置才熟悉,同时也从官方提供的手册中提取出较为常用的接口进行说明,最后将我熟悉的套路总结成文档回馈开源。
由于我水平有限,难免会存在理解错误的地方,欢迎读者发邮件指出,如果您有其他宝贵的建议,也欢迎发邮件交流,希望这个文档能帮助更多的开发者。

\section{本书结构}
\label{本书结构}

本书主要作为OMNeT++仿真平台快速入门指导书,书中大量来源于OMNeT++自带参考。从第一章到第五章,主要编写在OMNeT++上设计仿真程序,第七章
主要编写OMNeT++仿真数据统计方法,后续可增添我在OMNeT++平台下编写的相关辅助工具。各章内容如下:

\begin{itemize}
\item 第一章:OMNeT++简介

\item 第二章:OMNeT++安装以及相关仿真库的安装

\item 第三章:

\item 第四章:

\item 第五章:

\item 第六章:

\item 第七章:

\end{itemize}

\section{适合读者}
\label{适合读者}

本书适合所有正在看的人,否则你也不会读到这里。作为作者来说,我认为此书尤其适合以下读者:

\section{如何使用本书}
\label{如何使用本书}

\section{如何写作本书的}
\label{如何写作本书的}

本书采用排版简单的markdown编写的,中途遇到转换成pdf的问题,几度停下。本书记录的一些我对OMNeT++使用的心得,总结了部分相关的接口,但都是只是一些皮毛。

\subsection{封面}
\label{封面}

封面设置改了好几版,最开始打算使用卡通角色,最后感觉不好看,还是采用了一个简单的封面。封面中篆刻乃我近期仿刻的藏书印-“微风闲坐古松”,后期若有必要再对封面进行修改。

\chapter{版本变化}
\label{版本变化}

下面是每次的版本变化清单,你也可以自己在Github上查看。

\section{2018年2月22日-2月25日}
\label{2018年2月22日-2月25日}

一个比较完备的发行版,80页左右。

\begin{itemize}
\item 加强了书的格式,解决了中文字体显示的问题,加上了色彩,加上了图书推荐页。

\item 加强了书的格式,前言和致谢挪到了目录前,附录区别于正常章节。

\item 补全了Git。

\end{itemize}

\chapter{致谢}
\label{致谢}

感谢Kmtalexwang,Stephenhua对omnetppp\_zh.pdf的LaTeX排版的维护,以及Ericsyoung对chapter\_6.md仿真结果分析的添加。
