%---------------------------------------------------------------------------
\chapter{OMNeT++}

\begin{summary}
OMNeT++是一个网络仿真工具,支持以太网、无线网等协议仿真,同时提供友好的仿真界面以及3D显示。
\\
\end{summary}

\section{OMNeT++简介}

OMNeT++,一个基于eclipse开发套件的开源网络仿真工具,目前主要在高校实验室进行一些网络仿真测试,对一些算法进行对比,它可以供使用者进行完成以下开发:
\begin{itemize}
\item C/C++开发;
\item 网络仿真程序设计。
\end{itemize}
毫无疑问,基于<b>eclipse</b>的开发工具肯定能支持普通的<b>C/C++</b>工程。
另外,在<b>OMNeT++</b>上网络仿真设计领域的优势在于,它是一个开源的项目,对大量的网络模型都提供代码支持。但是问题在于国内的确没有什么社区支持,出现问题只能自己解决,其实对于开源的项目大多存在这种问题,往往开源的项目,使用起来难度较大,开源项目往往比那些商业的软件开发难度较大,支持也较少,开源可不代表简单。</br>
<b>OMNeT++</b>对初学者能力要求高,它假定使用者对编程有一定了解的,对eclipse开发环境也是特别熟悉的,另外这是一个网络仿真的软件,需要你对计算机网络有足够的认识,它提供了大量现有各种网络的仿真例子,如果你对网络认识足够强,那么这个软件你用起来会感到特别顺手。
\\
\section{我的初衷}

这是一个计划,计划在2018年写一个OMNeT++编程指导手册。

\begin{definition}
  由于我在学习使用\bm{$OMNeT++$}的过程中,遇到很多问题,虽然在安装的过程中,没有遇到什么阻碍,如果只是简单的想仿真自带的例子,估计就只需要修改仿真程序配置文件,在加上能分析仿真结果就行,但是想要完全自己写一个仿真程序,这些是完全不够的,这方面可以上YouTobe上搜索,一堆OMNeT++仿真的程序,基本都是自己开发的,而在国内,各论文上的OMNeT++仿真应该是自己写的,但是基本不会提供源代码。\newline
  我在后面的网络设计过程中,遇到很多疑问,每遇到一个问题都是花了两三个小时才解决,其中一些问题,也就是设置问题。特此总结一下我在OMNeT++里边踩的坑。
\end{definition}

\begin{remark}
  A remark may be in order here. This definition is concerned with
  \emph{rational} Cauchy sequences. We will later encounter a similar
  definition of \emph{real} Cauchy sequences.
\end{remark}

\begin{example}[Solving the equation $x^2 = 2$]
  Consider the equation $x^2 = 2$. It is easy to prove that this
  equation does not have any rational solutions. However, consider
  the following iteration formula:
  \begin{equation}
    x_n = \frac{x_{n-1} + 2 / x_{n - 1}}{2},
  \end{equation}
  where $n = 1,2,3,\ldots$ and $x_0 = 1$. The resulting sequence of
  rational numbers quickly approaches a number in the vicinity of
  $x = 1.4142135623731$:
  \begin{displaymath}
    \begin{array}{rclcl}
      x_0 &=& 1 \\
      x_{1} &=& (x_{0} + 2 / x_{0}) / 2 &=& 1.5 \\
      x_{2} &=& (x_{1} + 2 / x_{1}) / 2 &\approx& 1.4166666666667 \\
      x_{3} &=& (x_{2} + 2 / x_{2}) / 2 &\approx& 1.4142156862745 \\
      x_{4} &=& (x_{3} + 2 / x_{3}) / 2 &\approx& 1.4142135623747 \\
      x_{5} &=& (x_{4} + 2 / x_{4}) / 2 &\approx& 1.4142135623731 \\
      x_{6} &=& (x_{5} + 2 / x_{5}) / 2 &\approx& 1.4142135623731 \\
      x_{7} &=& (x_{6} + 2 / x_{6}) / 2 &\approx& 1.4142135623731 \\
      x_{8} &=& (x_{7} + 2 / x_{7}) / 2 &\approx& 1.4142135623731 \\
      x_{9} &=& (x_{8} + 2 / x_{8}) / 2 &\approx& 1.4142135623731 \\
      x_{10} &=& (x_{9} + 2 / x_{9}) / 2 &\approx& 1.4142135623731
    \end{array}
  \end{displaymath}
  We will later see that this iteration, or any other equivalent
  iteration, defines the real number $\sqrt{2}$.
\end{example}

\section{目录}

Now let's move on to the definition of the real number system. This
may be defined in a multitude of ways, one of which is to think about
a real number as a rational Cauchy sequence, or rather the equivalence
class of Cauchy sequences ``converging to'' that number.

\begin{definition}[The real numbers $\mathbb{R}$]
  \label{def:realnumbers}
  \index{real numbers}
  The real numbers $\mathbb{R}$ is the set of all equivalence classes
  of rational Cauchy sequences.
\end{definition}

Now that this is settled, lets prove the completeness of the real
number system.

\begin{theorem}[The completeness of the real numbers]
  \label{th:realnumberscomplete}
  \index{completeness of the real numbers}
  Let $(x_n)_{n=0}^{\infty}$ be a sequence of real numbers.
  Then $(x_n)_{n=0}^{\infty}$ is convergent if and only if
  it is also a real Cauchy sequence.
  \end{theorem}
\begin{proof}
  Write $x_m = [(x_{mn})_{n=0}^{\infty}]$ where
  $x_{mn}$ is the $n$th number in a rational Cauchy sequence
  representing the real number $x_m$. And so on\ldots.
\end{proof}

For further reading, there are several excellent works that one could
cite, such as \cite{Tao2006,Turing1936}.

\section*{Exercises}

\begin{exercise}
  Let $A = \{1, 2, 3\}$ and $B = \{2, 3, 4\}$.
  Determine the following sets. \\
  (a) $A \cup B$ \quad
  (b) $A \cap B$ \quad
  (c) $A \setminus B$ \quad
  (d) $A \times B$
\end{exercise}

\begin{exercise}
  Let $A = \{1, 3, 5, 7, 9\}$ and $B = \{2, 4, 6, 8, 10\}$.
  Determine the following sets. \\
  (a) $A \cup B$ \quad
  (b) $A \cap B$ \quad
  (c) $A \setminus B$ \quad
  (d) $A \times B$
\end{exercise}

\begin{exercise}
  Let $A = \{1, 2, 3\}$, $B = \{2, 3, 4\}$ and $C = \{3, 4, 5\}$.
  Determine the following sets. \\
  (a) $A \cup B \cup C$ \quad
  (b) $A \cap B \cap C$ \quad
  (c) $(B \setminus A) \cap C$ \quad
  (d) $(A \times B) \times C$
\end{exercise}

\section*{Problem}

\begin{problem}
  Interpret the following set definition (Russell's paradox) and discuss
  whether $X \in X$ or $X \notin X$:
  \begin{equation}
    X = \{x \mid x \notin x\}.
  \end{equation}
\end{problem}

\section*{Computer exercises}

\begin{programming}
  Write a program that generates the sequence $(x_n)_{n=0}^{100}$
  for $x_n = n$.
\end{programming}

\begin{programming}
  Write a program that generates the odd numbers between $1$ and $100$.
\end{programming}

\begin{programming}
  Write a program that computes the sum $\sum_{n=0}^{100} x_n$
  for $x_n = n$.
\end{programming}

%---------------------------------------------------------------------------


